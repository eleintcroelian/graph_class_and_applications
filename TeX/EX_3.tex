\documentclass[12pt,letterpaper,twoside]{article}

\newif\ifsolution\solutiontrue   % Include the solutions
%\newif\ifsolution\solutionfalse  % Exclude the solutions

\usepackage{cme212}

\newcommand{\T}[1]{\text{\texttt{#1}}}
\newcommand{\V}[1]{\text{\textit{#1}}}

\begin{document}

{\centering \textbf{Paper Exercise 3\\ Due Tuesday, February 25th at 4:30 P.M. P.S.T. \\}}
\vspace*{-8pt}\noindent\rule{\linewidth}{1pt}

\paragraph{Question 0: Lecture 5} Our graph is undirected, but we could use it to build a
\href{https://en.wikipedia.org/wiki/Directed_graph#Definition}{\emph{directed} graph} or \emph{digraph}. A digraph's edges are \emph{ordered}
pairs of nodes $(n_i, n_j)$, rather than unordered pairs $\{n_i, n_j\}$; all
operations look like their undirected counterparts and have similar complexity.

Here's one digraph representation using our graph class.
\begin{cpp}
template <typename V> 
class Digraph { 
  ... 
 private:
  Graph<V> g_;
};
\end{cpp}
Explain the representation by giving an abstraction function and
representation invariant. Note that \texttt{Digraph} is not
\texttt{Graph}'s friend and can only use \texttt{Graph}'s public
interface. \\

\textbf{Note:} To be concrete, we are asking: How can you use an undirected
graph to construct (or represent) a directed graph? This alludes to
the representation. Are there ``any
assumptions'' used during your construction? If so, these may be
necessary invariants.\\

\textbf{Note: } The representational invariants and the abstraction functions should not refer to the way you implemented your graph class but only to the abstract concept of an undirected graph.

\begin{solution}

\end{solution}

\paragraph{Question 1: Lecture 5} The \href{https://en.cppreference.com/w/cpp/container/vector/resize}{\texttt{vector<T>::resize}} operation extends a vector with
\emph{memory initialized} elements. (For example, after \texttt{vector<int> v;
  v.resize(20)}, \texttt{v} contains 20 zeroes in memory.) This is almost always what we
want, but it can be useful to extend a vector with \emph{uninitialized
  memory} so that we do not take up unnecessary space. Here's an example: uninitialized memory helps build a \emph{sparse
  vector}, in which elements are initialized only when first referenced. We
assume a special \texttt{garbage\_vector} type whose \texttt{resize} method does
not initialize new memory. \textbf{See next page}\\ \textbf{Note:} This question will
not completely make sense until you understand what the code is doing below!

\newpage
\begin{cpp}
template <typename T> 
class sparse_vector {
  garbage_vector<size_t> position_;
  vector<size_t> check_;
  vector<T> value_;
 public:
  // Construct a sparse vector with all elements equal to T().
  sparse_vector() {}
  // Return a reference to the element at position @a i.
  T& operator[](size_t i) {
    // If out of bounds, we must re-size our array and add (uninitialized) memory!
    if (i >= position_.size())
      position_.resize(i + 1);
    // If we haven't initialized the memory yet, go ahead and do so now.
    if (position_[i] >= check_.size() || check_[position_[i]] != i) {
      position_[i] = check_.size(); // First reference to the element at position i.
      check_.push_back(i);          // We associate an index...
      value_.push_back(T());        // ...alongside a value.
    }
    return value_[position_[i]];
  }
};
\end{cpp}

An abstract \texttt{sparse\_vector} value is an \textbf{infinite} vector.

Write an abstraction function and representation invariant for
\texttt{sparse\_vector}.

\begin{solution}
 
\end{solution}



\paragraph{Question 2: Lecture 12} The \texttt{CME212/BoundingBox.hpp} is kind of interesting. The following code will create a bounding box that encloses our entire graph:
\begin{cpp}
template <typename V, typename E>
Box3D graph_bounding_box(const Graph<V,E>& g) {
  auto first = graph.node_begin();
  auto last  = graph.node_end();
  assert(first != last);
  Box3D box = Box3D((*first).position());
  for (++first; first != last; ++first)
    box |= (*first).position();
  return box;
}
\end{cpp}
\textbf{This will be very important for homework 4.}
Interestingly, \texttt{Box3D} also provides a constructor that takes a range of \texttt{Point}s and performs the above operation for us. So, instead, we could write
\begin{cpp}
template <typename V, typename E>
Box3D graph_bounding_box(const Graph<V,E>& g) {
  return Box3D(g.node_begin(), g.node_end());
}
\end{cpp}
except that this doesn't work because we're passing it \texttt{Node} iterators instead of \texttt{Point} iterators.
The Boost and Thrust libraries both provide ``fancy iterators'', one of which is the \texttt{transform\_iterator}. See
\href{https://github.com/thrust/thrust/tree/master/thrust/iterator}{https://github.com/thrust/thrust/tree/master/thrust/iterator}
for documentation.

Use 
\href{https://github.com/thrust/thrust/blob/master/thrust/iterator/transform_iterator.h}{\texttt{thrust::transform\_iterator}} 
to correct the above code that uses \texttt{Box3D}'s range constructor.

\begin{solution}

\end{solution}

\paragraph{Question 3: Lecture 12} In fact, it is possible to implement your \texttt{Graph::node\_iterator} as a \texttt{thrust::transform\_iterator}. That is, we could delete \texttt{NodeIterator} entirely and write
\begin{verbatim}
using node_iterator = thrust::transform_iterator<__, __, Node>;
\end{verbatim}
The third parameter simply explicitly states that the \texttt{value\_type} should be \texttt{Node}. Fill in the two blanks and explain your answer.

\begin{solution}

\end{solution}

\paragraph{Question 4: Lecture 10} 

We may want to have multiple initialization methods for a class that basically do the same thing but have slightly different interfaces. It would be inefficient to duplicate the code in both constructors. Therefore since C++11 you can use \href{https://en.cppreference.com/w/cpp/language/initializer_list#Delegating_constructor}{delegate constructors}, where one constructor calls another.

Imagine we want to write a class to contain the scores of a test or assignment. The class will be constructed from a vector containing the scores. However depending on the assignment this might be a vector of letter grades or a vector of percentages. Instead of having to write two entire constructors we want to have one constructor delegate to another.

\textbf{(a)} Write a functor that takes in a letter grade from the list [A,B,C,D,E,F] and returns an int with the equivalent percentage of that grade.

\begin{solution}

\end{solution}

\textbf{(b)} We want to delegate to the constructor that takes in the start and end of a vector of percentages with the following interface.
\begin{cpp}
using intIter = std::vector<int>::iterator
using charIter = std::vector<char>::iterator
class Grades
{
	public:
	Grades(intIter start, intIter end){
		// normal construction - pretend this is already implemented
	}
}
\end{cpp}

Write a delegating constructor that takes in a \texttt{std::vector<char>} and your functor and delegates to the constructor above.

\begin{solution}

\end{solution}


\textbf{(c)} You could also get rid of duplicate code with an \texttt{init()} method that is called by multiple constructors. What is a potential issue with this implementation? 

\begin{solution}

\end{solution}
\end{document}




