\documentclass[12pt,letterpaper,twoside]{article}

\usepackage{cme212}

\begin{document}

{\centering \textbf{Paper Exercise 2\\ Due Tuesday, February 11th by
    4:30 P.M. PST via GradeScope\\}}
\vspace*{-8pt}\noindent\rule{\linewidth}{1pt}

\paragraph{Question 0: } Write an implementation of {\tt compute\_median} below that computes and returns the median value of an array of {\tt double}s. The implementation should have an average time complexity no worse than $\O(n)$ and are only allowed to use functionality available in the header files \href{https://en.cppreference.com/w/cpp/algorithm}{{\tt <algorithm>}} and \href{https://en.cppreference.com/w/cpp/container/vector}{{\tt <vector>}}. For an even number of values, return the mean of the two middle values. HINT: Find a more general function in {\tt <algorithm>} that you can map this {\tt compute\_median} onto.
\begin{cpp}
#include <algorithm>
#include <vector>

/** Compute the median of an array of doubles.
 * @param[in] values  The array of values.
 * @param[in]      n  The number of values in the array.
 * @result The median value of the array.
 *
 * @pre @a n > 0
 * @post Let sorted_v be @a values sorted by op<.
 *         If @a n odd,  result ==  sorted_v[n/2]
 *         If @a n even, result == (sorted_v[n/2] + sorted_v[n/2-1])/2
 */
double compute_median(const double* values, unsigned n)
\end{cpp}


\paragraph{Question 1: Lecture 4} Consider a data type $U$ with abstract type $U =
\{u_0,\dots,u_{n-1}\}$, where each $u_i$ is an integer. $U$'s iterator
iterates over the values in \emph{any} order. Assume $U$ has been implemented with a \texttt{std::vector<int> value\_} that contains all the elements. Write an
implementation of \texttt{U::iterator U::erase(U::iterator it)} that
has $O(1)$ time complexity. The specification is not necessary and we
emphasize that the returned iterator references an unordered
collection of elements.


\paragraph{Question 2: Lecture 4} Use the \texttt{erase} method you wrote above to write a method with the
following specification. Never use an invalid iterator---all iterators to a
\texttt{vector} become invalid after any \texttt{insert} or
\texttt{erase}---and your method should work for other STL containers, such
as \href{https://en.cppreference.com/w/cpp/container/map}{\texttt{map}} and 
\href{https://en.cppreference.com/w/cpp/container/list}{\texttt{list}}.

\begin{cpp}
/** Erase all elements of @a x for which @a pred returns true.
 * @param[in,out] x Container of elements.
 * @param[in] pred Predicate that takes an element and returns a bool.
 *
 * @post new @a x.size() <= old @a x.size()
 * @post new @a x contains exactly those elements e of old @a x for 
 *     which @a pred(e) returned false, and in the same relative order.
 */
template <typename Container, typename Predicate>
void erase_if(Container& x, Predicate pred);
\end{cpp}

\paragraph{Question 3: Lecture 6}
Remember the \texttt{Book} class from Exercise 1?
\begin{cpp}
	class Book:
	{
		public:
		Book(std::string filename) 
		{
			std::string sent;
			std::ifstream book_file(filename);
			while(std::getline(book_file, sent))
			{
				book_.push_back(split(sent, ' '));
			}
			book_file.close();
		}
		
	}
\end{cpp}

There is a problem with this class, it does not implement RAII. There are two points where we need to fix this.

\textbf{a}

In the constructor if anything goes wrong in the \texttt{while} loop the error will immediately propagate up the stack and the \texttt{close()} function will never be called. This can cause memory leak. Use exception handling to fix this issue.

\textbf{b}

Assume that we are reading in large books and are worried that we might run out of space on the stack so we are allocating \texttt{std::vector<std::vector<string>> book\_} on the heap. this would change the class outline as such.
	\begin{cpp}
	class Book:
	{
		public:
		Book(std::string filename) 
		{
			std::string sent;
			std::ifstream book_file(filename);
			try{
				while(std::getline(book_file, sent))
				{
					book_->push_back(split(sent, ' '));
				}
				book_file.close();
			}
			catch(){
				book_file.close()
				std::cout<<"something went wrong while reading the book."<<std::endl;
				throw;
			}	
		}
		private:
		std::vector<std::vector<string>> *book_;
	}
\end{cpp}

Now there is another problem. The default destructor does not clean up the heap allocated memory. This means that as soon as a book object goes out of scope we have a memory leak. Implement the \texttt{Book} class following RAII.

\paragraph{Question 4: Lectures 6 \& 7}
We have to be careful with resource allocation when it comes to inheritance. Consider the following polymorphism example.

\begin{cpp}
	class Student
	{
		private:
			std::string name_;
		public:
			Student(std::string name): name_(name){};
			virtual std::string print()
			{
				std::cout<<"Hello my name is "<< name_<<std::endl;
			}
		
	};
	
	class UndergradStudent: public Student
	{
		private:
		std::string uni_;
		public:
		Student(std::string name, std::string uni): Student(name), uni_(uni) {};
		
		virtual std::string print()
		{
			std::cout<<"Hello my name is "<< name_<<std::endl;
			std::cout<<"I am an undergrad at "<< uni_<<std::endl;
		}
		
	};
	
	class GradStudent: public Student
	{
		private:
		const fstream thesis_;
		
		public:
		GradStudent(std::string name, std::string uni, std::string thesis): 
		Student(name, uni)
		{
			thesis_.open(thesis);
		};
	
		virtual std::string print()
		{
			std::cout<<"Hello my name is "<< name_<<std::endl;
			std::cout<<"I am a grad student at "<< uni_<<std::endl;
		};
		
		~GradStudent(){
			thesis_.close();
		};	
	};
	
	int main()
	{
		std::vector<Student*> students;
		students.push_back(GradStudent('Amy','Stanford', 'Amys_thesis.txt'));
		students.push_back(Student('Karl','Berkley'));
		students.push_back(Student('Steve'));
		
		for(auto student= students.begin(); student != student.end(); student++)
		{
			student->print();
		}
		delete students;
	}
\end{cpp}

	What goes wrong in the code above and how do you fix it?
	

\end{document}

